\documentclass{article}
\usepackage[utf8]{inputenc}
\usepackage[T1]{fontenc}
\usepackage{polski}
\usepackage{indentfirst}
\usepackage{lastpage}
\usepackage{graphicx} 

\usepackage{fancyhdr}
\pagestyle{fancy}
\fancyhf{}
\rhead{Franciszek Wysocki, Bartosz Zdybel}
\lhead{Specyfikacja Funkcjonalna}
\rfoot{Strona \thepage \hspace{1pt} z \pageref{LastPage}}
\lhead{Spis treści}
\title{Specyfikacja Funkcjonalna dla projektu pt. \\ ,,Gra w życie''}
\author{}
\date{}

\begin{document}
\maketitle

\begin{flushright}
\par ...
\vfill
\par
Wykonali: Franciszek Wysocki, Bartosz Zdybel

Sprawdzający: mgr inż. Paweł Zawadzki

Data: 08.03.2020
\end{flushright}

\thispagestyle{empty}
\newpage
\begin{frame}{}
    \tableofcontents
\end{frame}
\newpage
\lhead{Opis teoretyczny}
\section{Opis teoretyczny}
{\fontsize{14}{14}\selectfont
Automat komórkowy - 
jest to uporządkowany zbiór komórek znajdujących się w jednym z dwóch stanach: żywym lub martwym.
Komórki znajdują się obok siebie, tworząc szachownicę. 
Każdy stan komórek w danej chwili nazywamy generacją. Możliwe jest stworzenie nowej generacji komórek według konkretnych zasad przy użyciu jednego z poniższych rodzajów sąsiedztwa:
\newline 
\newline
\begin{itemize}
\item Moore'a - sprawdza 8 sąsiadujących komórek znajdujących się obok sprawdzanej komórki oraz stykających się rogami (tylko szare i czarne komórki)

\begin{figure}[h]
\centering
\includegraphics[width=1.5cm]{Moor.png}
\caption{Sąsiedztwo Moore'a}
\label{fig:obrazek MOOR.png}
\end{figure}
\item von Neumanna - sprawdza 4 sąsiadujące komórki, ale tylko te, które stykają się bokami (tylko czarne kratki w rysunku poniżej)

\begin{figure}[h]
\centering
\includegraphics[width=1.5cm]{Neumann.png}
\caption{Sąsiedztwo von Neumanna}
\label{fig:obrazek Neumann.png}
\end{figure}
\end{itemize}

\begin{description}
\item[Gra w życie Johna Conwaya:]
\end{description}
Komórka może znajdować się w jednym z dwóch stanów:
\begin{itemize}
\item biały kolor (lub cyfra 0) - komórka martwa
\item czarny kolor (lub cyfra 1)- komórka żywa 
\end{itemize}
Zasada tworzenia nowej generacji komórek:
\begin{itemize}
\item jeśli martwa komórka ma 3 żywych sąsiadów to się rodzi
\item jeśli żywa komórka posiada 2 albo 3 żywych sąsiadów to pozostaje nadal żywa inaczej umiera z samotności albo przetłoczenia 
\end{itemize}
}
\lhead{Sposób wywołania programu}
\section{Sposób wywołania programu}
{\fontsize{14}{14}\selectfont 
W celu kompilacji można wywołać funkcję make,
a następnie w wierszu poleceń wpisać:
\newline
./a.out [nazwa pliku z danymi wejściowymi] [ewentualne flagi]
\newline
Przykładowe flagi (mogą być podane w dowolnej kolejności):
\begin{itemize}
\item -S - zapisuje generację do pliku - wymaga podania nazwy pliku jako kolejnego argumentu
\item -SBS - pokazuje każdą kolejną generację krok po kroku
\item -N - ustala ilość generacji (domyślnie jest jedna)  i tworzy N + 1 obrazów png (1 ze stanu poczatkowego i N ze stanów po kolejnych iteracjach) - wymaga podania liczby jako kolejnego argumentu
\end{itemize}
}

\section{Dane wejściowe}
{\fontsize{14}{14}\selectfont 
Użytkownik do programu będzie przekazywał plik, w którym będą znajdować się:
\newline
szerokość X wysokość rodzaj\_sąsiedztwa rodzaj\_generowania\_siatki
\newline
\newline
Przykładowe dane w pliku:
\newline
\texttt{10 X 10 MOOR RAND}
\newline
\newline
Istnieje jeszcze możliwość skorzystania z rodzaju generowania siatki \texttt{NORMAL}, natomiast wymaga to podania w dalszej części pliku macierzy o wymiarach szerokość na wysokość składającej się z 0 i 1 (w formie ciągów liczbowych).
\newline
\newline
Dozwolone rodzaje sąsiedztwa to RAND i NEUMANN.
\newline
\newline
Przykładowy dane w pliku z wykorzystaniem \texttt{NORMAL}:
\newline
\texttt{5 X 5 NEUMANN NORMAL}
\newline
\texttt{10110}
\newline
\texttt{10001}
\newline
\texttt{01010}
\newline
\texttt{10100}
\newline
\texttt{11100}
\newline

Wielkość liter będzie miała znaczenie przy podawaniu danych, także w przypadku podania ,,normal'' lub ,,rand'' w pliku, program się nie wykona i wyświetli stosowny komunikat.
}

\section{Dane wyjściowe}
{\fontsize{14}{14}\selectfont 
Po uruchomieniu programu użytkownik otrzyma informację o tym, na którym sąsiedztwie oparta jest gra, następnie w zależnośći od wyżej wymienionych flag zostanie wygenerowany tylko pierwszy i ostatni stan generacji, albo wszystkie generacje po kolei (-SBS). Zostaną również wygenerowane czarno-białe szachownice ilustrujące każdą kolejną generację w formacie ,,png''.}
\section{Działanie programu}
{\fontsize{14}{14}\selectfont 
Użytkownik podaje dane w wyżej wymieniony sposób. W przypadku podania złych danych program automatycznie się przerywa i podaje komunikaty o błędach.
\newline 
W przypadku podania prawidłowych danych wejściowych program wykona porządane rezultaty oraz zakończy pracę. }

\lhead{Komunikaty o błędach}
\section{Komunikaty o błędach}
{\fontsize{14}{14}\selectfont 
Program będzie odpowiednio zabezpieczony, a w przypadku błędów ze strony użytkownika będą wyświetlane komunikaty o:
\begin{itemize}
\item niemożliwej do zaalokowania pamięci
\item podaniu złych danych wejściowych w celu wywołania pliku
\item  nieudanym otwarciu pliku
\item  złym rozszerzeniu pliku, albo o błędnych danych znajdujących się w nim
\end{itemize}

}

\lhead{Testowanie programu}
\section{Testowanie programu}
{\fontsize{14}{14}\selectfont
W celu minimalizacji błędów oraz ułatwienia kolejnych kroków implementacyjnych zamierzamy do każdego modułu napisać program testowy, który ma na celu sprawdzenie czy program działa poprawnie.
Testy będą niezależne od działania innych modółów oraz będą automatycznie uruchomiane przy użyciu programu ,,make''.}
\end{document}